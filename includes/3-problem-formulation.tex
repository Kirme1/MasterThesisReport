\section{Problem Formulation}\label{sec:problem-formulation}

% The formulation of the problem at hand and, the assignment. This should include an extended version of the scientific problem definition and references to knowledge within the area given in the thesis proposal.

The problem to be solved with this thesis is that of efficiently compressing vehicle telemetry data for downstream machine learning tasks. The issue is that vehicles generate a vast amount of telemetry data from various sensors and systems, which can be very challenging to store and transmit due to constraints on edge devices and communication bandwidth \citep{samantaray2023}. While techniques for data compression do exist, they are either not optimized for the specific characteristics of vehicle telemetry data or do not take into account the requirements of downstream machine learning tasks.

Traditional compression techniques such as event-triggered logging and algorithmic time series compression methods have limitations when applied to vehicle telemetry data. Event-triggered logging can miss important information that does not cross predefined thresholds, leading to incomplete data for machine learning models \citep{nunes2023, jimenez2020, azar2022}. Algorithmic compression methods often rely on manually chosen parameters and may not effectively capture the complex patterns present in vehicle telemetry data, which can negatively impact the performance of downstream machine learning tasks. 

More recent approaches such as utility-aware adaptive telemetry and task-aware compression have shown promise in addressing some of these challenges. However, these methods often require complex policy learning or adaptive algorithms that may not be feasible for real-time applications in vehicles \citep{zheng2023, löhdefink2019, kawawabeaudan2022, liu2024deepdictdeeplearningbased}.

One less explored but promising approach to compressing data which could be suitable for vehicle telemetry data is that of neural compression in combination with tokenization. A system which incorporates a lightweight neural encoder and a small tokenizer should in theory be able to produce a lossy low entropy representation of the data which can be efficiently stored and transmitted. The main challenge will be to ensure that the compressed representation retains as much relevant information as possible for downstream machine learning tasks while still maintaining a high compression ratio.