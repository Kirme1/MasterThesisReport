\section{Ethical Concerns}\label{sec:ethical-concerns}

There are a few ethical concerns which might affect the project, although the concerns are relatively minor. It is also worth noting that the project does not involve human subjects which further reduces the ethical concerns. 

\textbf{Environmental Impact: } Since the project involves training several machine learning models on multiple datasets, the project will likely produce a non-negligible amount of carbon emissions. This issue is hard to address as we are forced to use computational resources in order to produce any meaningful results. However, we aim to minimize the environmental impact by using relatively small downstream machine learning models for testing the performance of the proposed compression method. This should produce informative results while keeping the computational requirements at a reasonable level. Another environmental concern is the use of neural compression over traditional compression methods, which are more computationally efficient. It should be acknowledge that the proposed method is likely to be more computationally intensive than traditional compression methods, and thus have a higher environmental impact. However, this will be mitigated, to the extent possible, by using a lightweight encoder and tokenization architecture. 

\textbf{Data Security: } The project will involve working with vehicle telemetry data from Volvo Group which might contain sensitive information. It is therefore important to ensure that the data is handled securely and in accordance with relevant data protection regulations. To mitigate this concern, all data will be kept internal to the research group and will not be explicitly shared in the final thesis document, only the produced results. To still allow the results to be reproducible and comparable, we will also use a publicly available dataset for evaluation and comparison. 

\textbf{Misuse of Technology: } The proposed compression method could potentially make a wide variety of systems more efficient, which if used in a malicious context could have negative consequences. For example, the compression method could be used to compress data in a way that makes it harder to detect malicious activity, such as cyber attacks or data breaches. While this is a valid concern, it is important to note that the proposed method is not designed to be intended for this type of use and that the potential for misuse is prevalent in every technology. To mitigate this it is important to emphasize that the intended use of the proposed method is to improve the efficiency of data storage and transmission in automotive embedded systems, and that any potential misuse of the technology is not condoned or supported.

It is also worth noting that the proposed method could be used in a more complex system which will make critical decisions based on the compressed data. In such a system it is important to acknowledge that the proposed framework could introduce biases and errors in the compressed data which could lead to incorrect decisions being made. To mitigate this concern, it is important to thoroughly evaluate the performance and output of the proposed method to ensure that it is accurate and reliable. Validating this will be a key part of the project and of high priority, but it is also worth noting that existing compression methods already introduce biases and errors in the compressed data, and that the proposed approach is not necessarily expected to be more prone to this than existing methods. 