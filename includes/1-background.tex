\section{Background}\label{sec:background}

% Definition of in-vehicle embedded system
\textbf{The In-Vehicle Embedded System:} An in-vehicle embedded system is a specialized computer system integrated within a vehicle to perform dedicated functions, often in real-time, and is essential for controlling, monitoring, and enhancing various automotive operations. These systems typically consist of both hardware and software components, such as electronic control units (ECUs), sensors, actuators, and communication interfaces, which are responsible for tasks like engine management, safety features, infotainment, and advanced driver assistance systems \citep{navet2017, fairley2019}.

% Justification for on board compression
\textbf{In-Vehicle Networks and Event-Triggered Logging:} Modern vehicles may contain dozens or even hundreds of these embedded systems, interconnected through in-vehicle networks (e.g., CAN, LIN, FlexRay, Ethernet), enabling efficient communication and coordination among different vehicle subsystems \citep{bello2019, navet2017, fairley2019}. Event-triggered logging and diagnostic frameworks, which record data only when anomalies or threshold crossings occur, are often adopted to reduce data transmission and avoid bus saturation in complex systems, such as the in-vehicle embedded system. However, this selective approach can reduce holistic visibility of system health, as it may miss subtle degradation patterns or early warning signs that do not cross predefined thresholds. This complicates the detection of emerging faults and comprehensive condition monitoring \citep{nunes2023, jimenez2020, azar2022}. Additionally, the need to carefully tune event thresholds and diagnostic criteria introduces maintenance challenges, as improper settings can lead to missed events or excessive false positives, further complicating system upkeep and reliability \citep{nunes2023, azar2022}.