\section{Risk Analysis}\label{sec:risk-analysis}

To prepare for potential delays and challenges in the project, we have identified several risks that could impact our timeline and deliverables. These risks are:
\begin{itemize}
    \item \textbf{Downstream model training takes longer than expected:} While there exist some fragments of downstream models within the Volvo environment, we will need to construct our own downstream models to test the compression methods. As this is not our main focus within this project, we hope to complete this relatively fast and with straightforward methods. But this, if e.g. data handling becomes more complicated than expected, this could become more time consuming than expected..
    \item \textbf{Baseline compression method is harder to implement than expected:} Fortunately \citep{zheng2023} describe the implemented method in detail, but there is still a risk, that our implementation of the baseline compression method may need finetuning and adjustments to work properly with our data and downstream models. This could lead to delays in the project timeline.
    \item \textbf{Proposed method does not outperform baselines:} There is a risk that the proposed tokenization-based compression method may not outperform the baseline methods in terms of compression efficiency or downstream task performance. This could lead to challenges in demonstrating the value of our approach and may require additional iterations or adjustments to the method.
    \item \textbf{Proposed method does outperform baseline, but is computationally not better:} Even if the proposed method achieves better compression or downstream performance, it remains to be seen, if the method actually outperforms the baseline in terms of computational efficiency. This could limit its practical applicability and may require further optimization or adjustments to make it viable for real-world use cases.
    \item \textbf{Computational efficiency cannot be proven without doubt because actual hardware is not available for testing:} As the actual in-vehicle technology will not be available for testing, the computational efficiency may be proven in theory not in practice. If the proposed method is actually computationally feasible in practice remains to be evaluated when it is deployed on actual hardware.
\end{itemize}   